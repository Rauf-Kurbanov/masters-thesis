\begin{frame}
\frametitle{\insertsection} 
\framesubtitle{\insertsubsection}
Для отладочных целей интерфейс Stream определяет метод peek.
С помощью него можем запомнить объекты перед и после вызова не изменив логику.
\begin{align*}
	&.peek(x \rightarrow store(x, time)) \\
	&.call(...)\\
	&.peek(z \rightarrow \{time.increment()\})\\
	&.peek(y \rightarrow store(y, time))
\end{align*}

В результате получим два множества $Before = \{(t_i, x_i)\}, After = \{(t_i, y_i)\}$. Они образуют состояния между вызовами.

Чтобы найти переходы достаточно построить \textbf{отображения}
\begin{align*}
	(t_i, x_i) \rightarrow List[(t_j, y_j)], \ \forall (t_i, x_i) \in Before \\
	(t_i, y_i) \rightarrow List[(t_j, x_j)], \ \forall (t_j, y_j) \in After
\end{align*}

\end{frame}
