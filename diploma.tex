% В этом шаблоне используется класс spbau-diploma. Его можно найти и, если требуется, 
% поправить в файле spbau-diploma.cls
\documentclass{spbau-diploma}
% test
\usepackage{wasysym }
\usepackage{subfiles}
\usepackage{enumitem}

% \usepackage[utf8]{inputenc}
% \usepackage[english]{babel}
 
% \usepackage{amsthm}

\newtheorem{theorem}{Теорема}[section]
\newtheorem{remark}{Замечание}[section]
\newtheorem{corollary}{Следствие}[section]
\newtheorem{lemma}[theorem]{Лемма}
\newtheorem{definition}{Определение}[section]
\newtheorem{assumption}{Предположение}
\newtheorem{observation}{Наблюдение}

% \usepackage{titlesec}
% \newcommand{\sectionbreak}{\clearpage}

\begin{document}
% Год, город, название университета и факультета предопределены,
% но можно и поменять.
% Если англоязычная титульная страница не нужна, то ее можно просто удалить.
\filltitle{ru}{
    chair              = {Кафедра математических и информационных технологий},
    title              = {Генерация речи с учётом индивидуальных особенностей},
    % Здесь указывается тип работы. Возможные значения:
    %   coursework - Курсовая работа
    %   diploma - Диплом специалиста
    %   master - Диплом магистра
    %   bachelor - Диплом бакалавра
    type               = {master},
    position           = {студента},
    group              = 604,
    author             = {Курбанов Рауф Эльшад оглы},
    supervisorPosition = {},
    supervisor         = {Шпильман А.\,А.},
    reviewerPosition   = {},
    reviewer           = {Тузова Е.\,А.},
    chairHeadPosition  = {д.\,ф.-м.\,н., профессор},
    chairHead          = {Омельченко А.\,В.},
    % university = {САНКТ-ПЕТЕРБУРГСКИЙ АКАДЕМИЧЕСКИЙ УНИВЕРСИТЕТ},
    % faculty = {Центр высшего образования},
    % city = {Санкт-Петербург},
    % year             = {2013}
}
\filltitle{en}{
    chair              = {Department of Mathematics and Information Technology},
    title              = {Speech generation with individual characteristics},
    author             = {Rauf Kurbanov},
    supervisorPosition = {},
    supervisor         = {Alexey Shpilman},
    reviewerPosition   = {},
    reviewer           = {Ekaterina Tuzova},
    chairHeadPosition  = {professor},
    chairHead          = {Alexander Omelchenko},
}
\maketitle
\tableofcontents

\section*{Введение}
\subfile{sections/introduction}

\subsection*{Мотивация}
\label{sec:motivation}
\subfile{sections/motivation}

\section{Обзор решений}
\label{sec:motivation}

\subfile{sections/existing_solutions}

\section{WaveNet}
\label{sec:wavenet}
\subfile{sections/wavenet}

\subsection{WaveNet с условием}
\label{sec:conditional_wavenet}
\subfile{sections/conditional_wavenet}

\section{Основные цели}
\label{sec:goals}
\subfile{sections/goals}


\section{Методы и реализация}

\subsection{Описание данных}
\label{sec:data}
\subfile{sections/data}

\label{sec:implementation}
\subfile{sections/implementation}

\newpage
\section{Обсуждение результатов}
\label{sec:experiments}
\subfile{sections/experiments}


\section{Заключение}
\label{sec:results}
\subfile{sections/results}

\bibliographystyle{ugost2008ls}
\bibliography{diploma.bib}
\end{document}
