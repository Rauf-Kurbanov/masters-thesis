% \documentclass[../diploma.tex]{subfiles}

% \begin{document}
        
    \label{sections/introduction}
    
    В данной работе исследуется метод генерации голоса с дополнительными характеристиками.
    Главным вдохновением к проведённой работе послужили недавние продвижения в области нейросетевых  генеративных архитектур, способных генерировать такие сложные вероятностные распределения как человеческая речь\cite{article:van2016wavenet}.
    
    Мы проведём обзор самых современных подходов к генерации голоса и сравним существующие решения с моделью на основе архитектуры WaveNet \ref{sec:existing_solutions}.
    
    Далее исследуем нюансы реализации легковесной модели для генерации голоса, показывающей state of the art качество голоса \ref{sec:wavenet}. А также поясним теоретическую базу модификации такой архитектуры для условной генерации \ref{sec:conditional_wavenet}. 
    
    Центральная часть работы посвящена деталям реализации упомянутой архитектуры а также проектированию и реализации признаков, передаваемых в качестве условий \ref{sec:implementation}.
    
    В заключении мы проведём эксперименты с разными конфигурациями модели и проанализирем результаты \ref{sec:experiments}.

% \end{document}

