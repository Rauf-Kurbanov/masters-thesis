% \documentclass[../diploma.tex]{subfiles}

% \begin{document}
        
    \label{sections/introduction}
    
    В данной работе исследуется метод генерации голоса с допольнительными характеристиками.
    Главным вдохновением к проведённой работе послужили недавние продвижения в области нейросетевых  генеративных архитектур, способных генерировать такие сложные вероятностные распределения как человеческая речь\cite{article:van2016wavenet}.

    % сделай отсылки на другие секции
    
    % Мне повезло начать писать свою магстрскую работу в эпоху второго рассвета нейросетей. Благодаря дешевизне вычислительных ресурсов нейроннные сети вновь привлекли внимание исследователей и инженеров и начали создавать вокруг себя отдельную подобласть машинного обучения. Сейчас разработка архитекут нейросетей стала обособленным искусством искусством. Каждый новый приём, придуманный архитекторами нейросетей открывает возможности для новый и новый state of the art решений, привлекая инеженеров и Data Scientist'ов. 
    
    % Одним из таких новыз приёмов, в каком-то смысле давших жизнь это статье стали дырявые свёртки. Такие свёртки дают возможность строить нейронные сети, способные справиться с крупномасштабным анализом входных данных с вполне приемлимой на практите вычислительной стоимостью. Грубо говоря, о слоях дырявых свёрток можно думать как о филтрах, позволяющих пропускать учаски данных, запонимая только "полезную" информацию и охватывающих очень большой участок.
    
    % Нельзя рассказать о вдохновении к этой работе, упустив краеугольную публикацию, на основе которой и строились все наши исслодования. Вот эта публикация непосредственно перенесла идею использования дырявых свёрток к контексте генератиыных моделей в таких работах как PixelCNN[ссыль] PixelNet[сслыль]. Авторы перенесли этот опыт в контекст генерации аудио в итоге разработав новый state of the art в генерации голоса WaveNet[сслыль].
    
    % В данной работе хотелось бы раскрыть потенциаль WaveNet, уточнить вскользь описанные аспекты реализации. В полной мере утилизировать концепт локальных услових в архитектуре. Во-вторых, после того-как получим детальную открытую и протестированную реализацию, хотелось бы внести модификации в архитектуру. 
    
    % Улитимативной целью было бы исслоедовать возможности feature engineering'a для передачи в сеть в качестве локальных условий и на основе последнних решения задачу text2speech c учётм индивидуальных особенностей голоса.
    
    % В чем идея? Ну я просто заценил классную статью про wavent и решил, но заметил что там написано ну оччень мало и было бы круто как минимум заимплеменитьт эту шутку и посмотреть как она работает. Если это и правда фурычит то это просто офигенно, пушечная нейросетка и даже не рекуррентная. Но потом мне пришла в голову идея ещё и прокачать эту штуку, потому что создатели заложили возможность использовать conditioning. То есть можно подавать допольнительные условия помимо звука. Тогад я решил что былы бы круто если нейросетть будет уметь призносить текст кастомным голосом того чувака, которого я захочу. 

% \end{document}

