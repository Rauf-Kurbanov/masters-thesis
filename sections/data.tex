\documentclass[../diploma.tex]{subfiles}
 
\begin{document}

В качестве данных для обучения в данной работе использовался корпус CSTR VCTK \cite{online:vctk}, размещённый в открытом доступе.
Корпус CSTR VCTK включает в себя речь, произнесённую 109 нативными носителями английского языка с разнообразными акцентами. 
Каждый оратор зачитывает около 400 предложений, выбранных из газет, также специальные отрывки, подобранные чтобы подчеркнуть акцент говорящего. 

К каждой голосовой записи прилагается текст, не выровненный по аудио. То есть каждый пример из обучающегося множества представлен в виде пары (аудио, текстовый файл). 
Аудио прдесавленно в формате .wav с частотой сэмплирования 48000 Гц и одним каналом, то есть моно. Текст - обычный .txt файл без выравнивания по времени, в противном случае это были бы субтитры.

Все отрывки речи были записаны с использованием одинакового оборудования, предоставленного университетом Эдинбурга. Также использовалось специальное помещение, поэтому качество записи достаточно высокое и практически не требует фильтрации шумов.

Продолжительность аудио варьируется от 1 до 11 секунд со стандартным отклонением около секунды, что более чем достаточно для нашей задачи, так как генерация даже небольших звуковых файлов занимает достаточно много машинного времени в силу скромности наших вычислительных ресурсов.

\end{document}

