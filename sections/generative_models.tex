\documentclass[../diploma.tex]{subfiles}

\begin{document}

В статистике и машинном обучении генеративная модель это модель для генерации данных при наличие скрытых параметров. Модель определяет совместное распределение вероятности над пространством наблюдений. Генеративные модели используются в машинном обучении либо для генерации данных напрямую, либо в качестве промежуточного шага для получения плотности распределения функции условной вероятности. Условное распределение может быть получено из генеративной модели с помощью теоремы Байеса.

Генеративные модели противопоставляются дискриминативным моделям в том плане, что генеративная описывает полную вероятностную модель для всех переменных, когда дискриминативная в свою очередь моделирует лишь вероятностное распределение лишь целевой переменной. Таким образом генеративная модель может быть использована для симуляции значений любой переменной, используемой в модели, когда дискриминативная дает возможность лишь получить целевую переменную по наблюдаемым признакам. Несмотря на то, что дискриминативным моделям не требуется симулировать распределение наблюдаемой переменной, они зачастую не могут описать более сложные отношения между признаками и целевой переменной. Такие модели не обязательно показывают лучше результаты в задачах классификации и регрессии. В современных приложениях данные два класса моделей зачастую служат разными подходами к одной и той же процедуре.
\cite{wiki:generative}

\end{document}

