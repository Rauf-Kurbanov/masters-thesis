\documentclass[../diploma.tex]{subfiles}
 
\begin{document}

Каких результатов мы добились:
% \subsection{Чего мы добились}

\begin{enumerate}
    \item Реализован WaveNet максимально придерживаясь описания из статьи.
    \begin{itemize}
        \item Доработана существующая генерация голоса без условия.
        \item Реализована генерация голоса по тексту.
        \item Реализована генерация голоса по тексту с условием.
    \end{itemize}    
\end{enumerate}

Мы дополнили публичную реализацию WaveNet с максимальной точностью повторяя описание из статьи, а также реализовали нюансы, опущенные в оригинальной статье, но важные для удобства использование модели. Одним из таких нюансов стало выравнивание длин локальных условий.

\begin{enumerate}[resume]
    \item Разработаны признаки для генерации голоса.
\end{enumerate}

Основным достижением в этом пункте мы считаем реализацию и апробацию набора признаков, основывающихся на сгенерированной речи худшего качества сгенерированной другим инструментом. Таким образом мы не только нашли легковесное высокоуровневое представление для локальных условий, но и неявно предоставили механизм для комбинации WaveNet с другими системами для генерации речи. 

\begin{enumerate}[resume]
    \item Получены результаты генерации.
\end{enumerate}

Мы получили порядка двадцати примеров сгенерированной речи на базе нашей модели с разными конфигурациями и сделали выводы о применимости рассмотренных подходов.

\subsection{Направления развития}

\begin{enumerate}
    \item Приспособить модель для решения задачи text2speech.
\end{enumerate}
Теперь, когда у нас есть полностью модифицированный WaveNet с некоторым каскадом дополнительных признаков и локальных условий, мы можем нацелиться на задачу генерации речи по тексту. В таком контексте стоит обратить больше внимания текстовым признакам и сместить центр внимания от натуральности генерируемой к  способности модели произносить тексты.

\begin{enumerate}[resume]
    \item Реализовать более ярко выраженную условную генерацию.
\end{enumerate}
В рамках экспериментов не удалось бесплатно достичь значимой разницы в речи при условной генерации за счёт включения простых глобальных условий. Основной причиной тому послужило падение качества речи при включении глобальных условий. Исходя из этого, у работы есть ещё и такой вектор развития в сторону более яркого контраста на базе глобальных условий.

\end{document}

