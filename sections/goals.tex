\documentclass[../diploma.tex]{subfiles}
 
\begin{document}

Сформулируем цели, которые мы ставим в этой работе:

\begin{enumerate}
    \item Реализовать WaveNet максимально придерживаясь описания из статьи.
    \begin{itemize}
        \item Реализовать генерацию голоса без условия.
        \item Реализовать генерацию голоса по тексту.
        \item Реализовать генерацию голоса по тексту с условием.
    \end{itemize}    
\end{enumerate}

Важно отметить, что ключевым аспектом этой подзадачи является педантичная точность в реализации архитектуры сети.
В оригинальной статье архитектура описана довольно высокоуровнево и в общих словах, что подразумевает от читателя уверенных знаний в области и навыков проектирования сетей.

Нам также требуется реализовать дополнительные модификации для локальных и глобальных условий, поскольку они необходимы для генерации с особенностями.

\begin{enumerate}[resume]
    \item Разработать признаки для генерации голоса
\end{enumerate}

Требуется спроектировать, реализовать и провести апробацию набора признаков, передаваемых в сеть по каналам локальных и глобальных условий. Авторы WaveNet бегло упоминают используемые признаки, оставляя эту задачу пользователям. 

\begin{enumerate}[resume]
    \item Получить результаты генерации
\end{enumerate}

Наша постановка целей сформулирована так, что мы не можем описать качество работы модели в виде численных измерений.
Поэтому постараемся добиться высоких качеств натуральности голоса в результате экспериментов и получить примеры сгенерированной речи.

\end{document}

