\documentclass[../diploma.tex]{subfiles}
 
\begin{document}

% Теперь мы знаем, как бы мы хотели видеть систему в идеале, осознаем немножко сложность и масштаб области.
% Можно сформулировать задачи, которые я планирую решить в рамках данной работы. 
Сформулируем цели, которые мы ставим в этой работе:

\begin{enumerate}
    \item Реализовать WaveNet максимально придерживаясь описания из статьи.
    \begin{itemize}
        \item Реализовать генерацию голоса без условия.
        \item Реализовать генерацию голоса по тексту.
        \item Реализовать генерацию голоса по тексту с условием.
    \end{itemize}    
\end{enumerate}

Важно отметить, что ключевым аспектом этой подзадачи является педантичная точность в реализации архитектуры сети.
В оригинальной статье архитектура описана довольно высокоуровнево и в общих словах, что подразумевает от читателя уверенных знаний в области и навыков проектирования сетей.

Нам также требуется реализовать дополнительные модификации для локальных и глобальных условий, поскольку они необходимы для генерации с особенностями.

% На самом деле проведя 10 минут в гугле, можно найти большой репозиторий с реализацией WaveNet, вокруг которое даже успело образоваться небольшое комьюнити. У нас ещё такая специфичаная задача, что надо воссоздать архитектуру максимально точно так, как подразумевали создатели, при том, что описали её достаточно высокоуровнево.

% Дело в том, что даже эта реализация далеко не серебянная пуля, так как наша работа очень сильно завиит от того, реализованы ли в сети дополнительные слои, отвечающие за локальный и глобальные условия
% Глобальных условий ещё не было на момент начала работы, а локальные в общем репозитории не появились до сих пор. Поэтому мне их надо реализовать максимально корректно.

\begin{enumerate}[resume]
    \item Разработать признаки для генерации голоса
\end{enumerate}

Требуется спроектировать, реализовать и провести апробацию набора признаков, передаваемых в сеть по каналам локальных и глобальных условий. Авторы WaveNet бегло упоминают используемые признаки, оставляя эту задачу пользователям. 

\begin{enumerate}[resume]
    \item Получить результаты генерации
\end{enumerate}

Наша постановка целей сформулирована так, что мы не можем описать качество работы модели в виде численных измерений.
Поэтому постараемся добиться высоких качеств натуральности голоса в результате экспериментов и получить примеры сгенерированной речи.

% Ну и в конце нужно как-то оценить качественно нащей модели и получить какие-то практические результаты. Как мы увидим после, даже на самом высоком уровне модель содержит много метапараметров.
% Если вдобавок к этому всмопнить, что для хорошей постановки эксперимента нужно подготовить правльний корпус данных да ещё и потратить много машинного времени, в силу высокой требовательнсоти по вычислительным ресурам, то это высрастает в отдельную задачу. 
% Путь не такую сложную интеллектуально или механически как предыдущие две, но требующую много времени и аккуртаности.

\end{document}

