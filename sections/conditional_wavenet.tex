\documentclass[../diploma.tex]{subfiles}
 
\begin{document}

Если дан допольнительный вход $h$ как условие, WaveNet может моделировать условное распределение $p(x|h)$ аудио по этому входу.
$$p(x|h) = \prod^{T}_{t=1}{p(x_{t}|x_1, \dots, x_{t-1}, h)}$$
Задавая такого рода условия, мы сподвигаем WaveNet к генерации аудио с необходимыми характеристиками. Например, если наши исходные данные для обучения содержает большое количество ораторов, мы можем передавать id оратора в качестве глобальго условия. Тогда впоследсвтие при генерации мы сможем выбирать, речь какого оратора мы хотим сгененрировать. 

\begin{itemize}
    \item Сырые данные
    \begin{itemize}
        \item Временной ряд, цифровое представление голоса по времени
    \end{itemize}
    \item Локальное условие 
    \begin{itemize}
        \item Временной ряд, той же длины что и данные. Качество, изменяющееся по времени
    \end{itemize}
    \item Глобальное условие
    \begin{itemize}
        \item Качество говорящего, не зависящее от времени. Не меняет своего значения в процессе обучения/генерации
    \end{itemize}
\end{itemize}


\subsection{Локальные и глобальные условия}
Архитектура WaveNet предоставляет два способа передачи условий: глобальное условие и локальное условие. Глобальное условие характризуется единственным скрытым предствалением $\bf{h}$, которое оказывает влияние на финальное распределение вдоль всего временного промежутка. Функция активации из формулы [ссыль] теперь принимает вид:
$$
z = \tanh(W_{f,k} * x + V_{f,k}^T h~\astrosun~\sigma (W_{g,k} * x + V_{g,k}^T h)).
$$
где $V_{*,k}$ это свёртка $1 \times 1$. В качестве альертантивы такой миниатюрной свёрточной сети можно использовать $V_{f,k} * h$ и продублировать это значения по времени. Авторы утрвеождают, что такое подход работает гораздо хуже на практике.  


\end{document}

